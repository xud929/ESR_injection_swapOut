\documentclass{article}
\usepackage{booktabs}
\usepackage{upgreek}

\usepackage[english]{babel}
\usepackage[letterpaper,top=2cm,bottom=2cm,left=3cm,right=3cm,marginparwidth=1.75cm]{geometry}
\usepackage{amsmath}
\usepackage{graphicx}
\usepackage[colorlinks=true, allcolors=blue]{hyperref}

\title{Study of Electron Transverse Emittance Mismatch in the EIC Swap-Out Injection Scheme}
\author{Derong Xu}

\begin{document}
\maketitle

\begin{abstract}
\textcolor{red}{Your abstract.}
Just to follow up on our conversations this morning, we would like to understand the feasibility of the following two concepts:

    Direct injection of 7-14 nC bunches from a 2-pass 5-GeV recirculator (2 x 2.5 GeV) into the ESR.  What are the possible emittances, momentum spread, bunch length at the ESR injection.
    Direct injection of 7-14 nC bunches at 10 GeV (3 x 3.3 GeV) into the ESR.

Assuming a racetrack tunnel with only one linac, we need to answer the following questions:

    What is the minimum arc radius we need to obtain emittances acceptable for the ESR injection? -- Todd
    What is the maximum bunch length we can obtain at injection and what is the minimum momentum spread? -- Todd
    Can we do a swap-out injection at 5 and 10 GeV? Wha is the proton emittance increase per one such injection into the ESR? -- Derong
    Can we accumulate 28 nC bunches in the ESR (4 x 7 nC) at 200 ms per injection? -- Derong
    What are the ESR beam losses per injection? Where are these particles being lost? – TBD
Erdong can provide bunch parameters after the LEBT and also after the first linac pass for 7 – 14 nC.
\end{abstract}
%%%%%%%%%%%%%%%%%%%%%%%%%%%%%%%%%%%%%%%%%%%%%%%%%%%%%%%%%%%%%%%
\section{Introduction}
The Electron-Ion Collider (EIC) is designed to achieve a high peak luminosity of 
$10^{34}~\mathrm{cm}^{-2}\mathrm{s}^{-1}$ through collisions of polarized electron and 
proton beams. The Electron Storage Ring (ESR) is expected to 
deliver high-charge electron bunches of up to $28~\mathrm{nC}$.
The ESR lattice is engineered to provide a dynamic aperture of $10\sigma$ in all three 
planes. Due to the limited polarization lifetime, frequent electron bunch replacement 
is required. The swap-out injection scheme offers an efficient solution to meet these 
demands, while accommodating the small dynamic aperture. The Rapid Cycling Synchrotron 
(RCS) is responsible for electron accumulation, acceleration, and injection to the ESR.

However, the design emittances of the RCS and ESR differ significantly. 
In the RCS, the vertical emittance is nearly zero to suppress vertical intrinsic 
spin resonances. In contrast, the ESR maintains a finite vertical emittance to 
match the proton beam size at the interaction point (IP). 
Although the electron beam can be manipulated during transport from the RCS to 
the ESR, maintaining precise control over the resulting emittance is challenging.
This emittance mismatch during electron injection can lead to proton emittance growth.

An alternative injection scheme involves injecting electron bunches directly from the 
LINAC into the ESR, eliminating the need for the RCS. This approach meets the Phase I 
goals, where the ESR only needs to provide $5/10~\mathrm{GeV}$ electron bunches with 
a maximum charge of $7~\mathrm{nC}$ \cite{EICMACReview202409:Sergei} . In Phase II, where a $28~\mathrm{nC}$ electron 
bunch is required, multiple electron bunches can be injected into the same ESR bucket. 
Off-momentum injection can be employed to minimize electron emittance blow-up, 
making this accumulation scheme feasible. Our previous study indicates that 
the accumulation scheme is a viable option \cite{xu:ipac2024-mopc72}.

The electron bunch from the LINAC also differs from the ESR design value. 
In this note, we apply the strong-strong simulation method to study proton 
emittance growth during the swap-out injection for electron bunch charges of $7$, 
$14$, and $28~\mathrm{nC}$, respectively. The electron injection emittance spans 
a wide range to account for both LINAC and RCS cases. The accumulation scenario 
with significant emittance mismatch will be revisited in future studies.
%%%%%%%%%%%%%%%%%%%%%%%%%%%%%%%%%%%%%%%%%%%%%%%%%%%%%%%%%%%%%%%
\section{Method}
To model the distribution evolution of both the electron and proton beams, 
a self-consistent simulation is required. Accordingly, we employ 
the strong-strong simulation method. The beam parameters used in the simulation 
are presented in Table~\ref{tab:SimulationParameters}. 

\begin{table}
  \centering
  \caption{Beam parameters used in strong-strong simulation. The columns ``Proton 
  design'' and ``Electron design'' are directly taken from EIC-CDR 
  \cite{willeke2021electron}. The ``Electron input'' is what we actually use in
  the simulation. Without beam-beam interaction, the electron beam parameters
  will evolve toward the ``Electron design'' parameter.
  ``H'' stands for horizontal and ``V'' denotes vertical below.
  Parameters marked in red as ``TBD'' indicate values that were scanned during 
  the simulation.\\}
  \begin{tabular}{lcccc}
   \toprule
    Parameter & Unit & Proton design & Electron design & Electron input\\
    \hline
    Circumference & $\mathrm{m}$ & $3834$ & $3834$ & $3834$ \\
    Energy & $\mathrm{GeV}$ & $275$ & $10$ & $10$\\
    Particles per bunch & $10^{11}$ & $0.688$ & $1.72$ & $1.72$\\
    Half crossing angle & $\mathrm{mrad}$ & $12.5$ & $12.5$ & $12.5$\\
    Crab cavity frequency & $\mathrm{MHz}$ & $200.0/400.0$ & $400.0$ & $400.0$\\
    $\beta_x^*/\beta_y^*$ & $\mathrm{cm}$ & $80.0/7.20$ & $55.0/5.60$ & $55.0/5.60$\\
    RMS emittance (H/V) & $\mathrm{nm\cdot rad}$ & $11.3/1.00$ & $20.4/1.6$ & \textcolor{red}{TBD}\\
    RMS bunch size (H/V) & $\mathrm{\upmu m}$ & $95.0/8.5$ & $106/9.5$ & \textcolor{red}{TBD}\\
    RMS bunch length & $\mathrm{cm}$ & $6.0$ & $0.7$ & $0.09$\\
    RMS energy spread & $10^{-4}$ & $6.6$ & $5.5$ & $20.0$\\
    Transverse fractional tune (H/V) & - & $0.228/0.210$ & $0.08/0.14$ & $0.08/0.14$\\
    Synchrotron tune  & - & $-0.010$ & $-0.069$ & $-0.069$\\
    Transverse damping time & turns & $\infty$ & $4000$ & $4000$\\
    Longitudinal damping time & turns & $\infty$ & $2000$ & $2000$\\
    Chromaticity (H/V) & - & $2/2$ & $1/1$ & $1/1$\\
  \bottomrule
  \end{tabular}
  \label{tab:SimulationParameters}
\end{table}

The horizontal emittance of the electron beam is scanned from $5~\mathrm{nm}$ to 
$23~\mathrm{nm}$ in steps of $2~\mathrm{nm}$. The vertical emittance is scanned 
from $0.5~\mathrm{nm}$ to $5.0~\mathrm{nm}$, with a step size of $0.5~\mathrm{nm}$. 
The injected electron bunches are assumed to match the ESR optics, and their initial 
beam sizes are determined based on the scanned emittance and beam optics. 
The initial proton beam follows a perfect Gaussian distribution and begins interacting 
with the electron beam from the 1st turn. After $50,000$ turns, the electron bunch 
is swapped out and replaced with a fresh one, allowing the HSR bunch to 
continue interaction with the newly injected electron bunch.

The strong-strong simulation is influenced by numerical noise, which varies based on 
the model parameters. The chosen model parameters are listed in Table \ref{tab:model}.
The ratio of macro particles is set equal to the ratio of bunch intensities, 
following our previous study. To reduce computation time, we also reduce the 
number of longitudinal slices and transverse grids.
\begin{table}
    \centering
    \caption{Model parameters during the strong-strong simulation\\}
    \begin{tabular}{lcc}
    \toprule
    Model parameter & Proton beam & Electron beam\\
    \midrule
    Number of macro particles & $512,000$ & $1,280,000$\\
    Number of longitudinal slices & $11$ & $11$\\
    Number of transverse grids & \multicolumn{2}{c}{$64\times 64$}\\
    \bottomrule
    \end{tabular}
    \label{tab:model}
\end{table}

%%%%%%%%%%%%%%%%%%%%%%%%%%%%%%%%%%%%%%%%%%%%%%%%%%%%%%%%%%%%%%%
\section{Result}

%%%%%%%%%%%%%%%%%%%%%%%%%%%%%%%%%%%%%%%%%%%%%%%%%%%%%%%%%%%%%%%
\section{Discussion}

%%%%%%%%%%%%%%%%%%%%%%%%%%%%%%%%%%%%%%%%%%%%%%%%%%%%%%%%%%%%%%%
\section{Summary}

%%%%%%%%%%%%%%%%%%%%%%%%%%%%%%%%%%%%%%%%%%%%%%%%%%%%%%%%%%%%%%%
\bibliographystyle{unsrt}
\bibliography{reference}

\end{document}